\documentclass{report} % clase de documento
% \documentclass{article} % default
\usepackage{algpseudocode} % Para escribir algoritmos
% paquetes necesarios
\usepackage{fancyhdr} % Encabezado y pie de página
\usepackage{geometry} % Márgenes del documento
\usepackage[numbers]{natbib} % Bibliografía con números
\usepackage[utf8]{inputenc} % Codificación de caracteres
\usepackage[spanish]{babel} % Idioma del documento
\usepackage{amsmath} % Paquete para escribir fórmulas matemáticas
\usepackage{graphicx} % Paquete para incluir imágenes
\usepackage{tikz} % Paquete para dibujar en LaTeX
\usepackage{pdfpages} % Paquete para incluir archivos PDF
\usepackage[colorlinks, linkcolor=black]{hyperref} % Paquete para hipervínculos
% \usepackage{lipsum} % Paquete para generar texto de ejemplo
\usepackage{tikz} % Paquete para dibujar en LaTeX
\usepackage{pgfplots} % Para gráficos 3D
\pgfplotsset{compat=1.18} % Compatibilidad con versiones recientes de pgfplots
\usetikzlibrary{automata, positioning} % Para dibujar autómatas
\usepackage{xcolor} % Para definir colores personalizados
\usepackage{fix-cm} % Para cambiar el tamaño de la fuente
% configuración de los hipervínculos del documento (colores)
\hypersetup{
    colorlinks=true, % false: boxed links; true: colored links
    linkcolor=black, % color of internal links
    citecolor=black, % color of links to bibliography
    filecolor=black, % color of file links
    urlcolor=black   % color of external links
}

\usetikzlibrary{automata, positioning, arrows} % Para dibujar autómatas

\geometry{margin=1in} % Márgenes de las páginas

% configuración de los encabezados y pies de página
\pagestyle{fancy} % Encabezado y pie de página
\fancyhead{} % Limpia el encabezado
\fancyhead[L]{Leonardo Daniel Aviña Neri} % Encabezado a la izqui
\fancyhead[R]{UNIVERSIDAD DE GUANAJUATO} % Encabezado a la derecha

\begin{document} % Inicio del documento

\begin{titlepage} % Portada

    % \begin{tikzpicture}[remember picture, overlay]
    %     \fill[blue!10,opacity=0.5] (current page.north west) rectangle ([yshift=-\paperheight/1]current page.north east);
    % \end{tikzpicture}

    \begin{flushleft}
        \includegraphics[width=0.4\textwidth]{../../img/logo_ug.png} % Logo de la universidad
    \end{flushleft}

    \begin{flushright}
        \vspace{-1.6cm}
        \textbf{UNIVERSIDAD DE GUANAJUATO}
    \end{flushright}

    \vspace{2cm}

    
    \begin{center}
        \huge{
            \textbf{TAREA 17}
        }
         % Título en negritas y muy grande
    \end{center}

    \vspace{2cm}
    
    \begin{flushleft}
        \hspace*{4cm}
        \textbf{UDA: }Matemáticas Discretas \\
        \hspace*{4cm}
        \textbf{DOCENTE: }M. I. Robles Cervantes Uriel  \\
        \hspace*{4cm}
        \textbf{ALUMNO: }Leonardo Daniel Aviña Neri \\
        \hspace*{4cm}
        \textbf{FECHA DE ENTREGA: } 25/05/2025 \\
        \hspace*{4cm}
        \textbf{CARRERA:} Licenciatura En Ingeniería De Datos \\
        \hspace*{4cm}
        E Inteligencia Artificial (LIDIA) \\
        \hspace*{4cm}
        % \textbf{CLAVE UDA:} NELI04245
    \end{flushleft}

    \vfill

    \begin{center}
        \includegraphics[width=0.4\textwidth]{../../img/logo_grande.jpg} \\ % Imagen centrada al final de la hoja
        \vspace{0.3cm}
        Semestre Enero-Junio \the\year \\
        \textbf{Campus Irapuato-Salamanca}
    \end{center}
\end{titlepage}


% \includepdfmerge[pages=-]{dis-17}

\section*{1.\ Calcule los siguientes determinantes por el método de cofactores.}

\subsection*{Primer determinante}
Calcularemos el determinante de la matriz:
\begin{align*}
\mathbf{A} = \begin{vmatrix}
2 & -3 & 0 & 0 \\
-2 & -2 & -2 & 0 \\
0 & -5 & 4 & 6 \\
-3 & 2 & -8 & 0
\end{vmatrix}
\end{align*}

Desarrollaremos el determinante por la última columna, que tiene dos ceros:

\begin{align*}
\det(\mathbf{A}) &= 0 \cdot C_{14} + 0 \cdot C_{24} + 6 \cdot C_{34} + 0 \cdot C_{44}\\
&= 6 \cdot C_{34}
\end{align*}

Donde $C_{34} = (-1)^{3+4} \cdot M_{34} = (-1)^7 \cdot M_{34} = -M_{34}$

El menor $M_{34}$ es:
\begin{align*}
M_{34} = \begin{vmatrix}
2 & -3 & 0 \\
-2 & -2 & -2 \\
-3 & 2 & -8
\end{vmatrix}
\end{align*}

Desarrollamos este menor por la tercera columna:
\begin{align*}
M_{34} &= 0 \cdot (-1)^{1+3} \begin{vmatrix}
-2 & -2 \\
-3 & 2
\end{vmatrix} + (-2) \cdot (-1)^{2+3} \begin{vmatrix}
2 & -3 \\
-3 & 2
\end{vmatrix} + (-8) \cdot (-1)^{3+3} \begin{vmatrix}
2 & -3 \\
-2 & -2
\end{vmatrix} \\
&= -2 \cdot (-1) \cdot [(2 \cdot 2) - (-3 \cdot -3)] + (-8) \cdot [(2 \cdot -2) - (-3 \cdot -2)] \\
&= 2 \cdot [4 - 9] - 8 \cdot [-4 - 6] \\
&= 2 \cdot (-5) - 8 \cdot (-10) \\
&= -10 + 80 \\
&= 70
\end{align*}

Por lo tanto:
\begin{align*}
\det(\mathbf{A}) &= 6 \cdot C_{34} = 6 \cdot (-70) = -420
\end{align*}

\subsection*{Segundo determinante}
Para la matriz:
\begin{align*}
\mathbf{A} = \begin{vmatrix}
1 & 2 & 1 & -2 & 0 \\
3 & 2 & -4 & 6 & 3 \\
0 & 2 & 0 & 6 & 2 \\
-3 & 2 & 0 & 0 & -2 \\
4 & 2 & 0 & -1 & -2
\end{vmatrix}
\end{align*}

Desarrollaremos por la tercera columna que tiene tres ceros:
\begin{align*}
\det(\mathbf{A}) &= 1 \cdot C_{13} + (-4) \cdot C_{23} + 0 \cdot C_{33} + 0 \cdot C_{43} + 0 \cdot C_{53} \\
&= 1 \cdot C_{13} - 4 \cdot C_{23}
\end{align*}

Calculemos $C_{13}$:
\begin{align*}
C_{13} &= (-1)^{1+3} \begin{vmatrix}
3 & 2 & 6 & 3 \\
0 & 2 & 6 & 2 \\
-3 & 2 & 0 & -2 \\
4 & 2 & -1 & -2
\end{vmatrix} = -1 \cdot \begin{vmatrix}
3 & 2 & 6 & 3 \\
0 & 2 & 6 & 2 \\
-3 & 2 & 0 & -2 \\
4 & 2 & -1 & -2
\end{vmatrix}
\end{align*}

Factorizando la columna 2 que tiene todos los elementos iguales a 2:
\begin{align*}
\begin{vmatrix}
3 & 2 & 6 & 3 \\
0 & 2 & 6 & 2 \\
-3 & 2 & 0 & -2 \\
4 & 2 & -1 & -2
\end{vmatrix} &= 2 \cdot \begin{vmatrix}
3 & 1 & 6 & 3 \\
0 & 1 & 6 & 2 \\
-3 & 1 & 0 & -2 \\
4 & 1 & -1 & -2
\end{vmatrix}
\end{align*}

Restando la columna 2 de las otras columnas para crear más ceros:
\begin{align*}
2 \cdot \begin{vmatrix}
2 & 1 & 5 & 2 \\
-1 & 1 & 5 & 1 \\
-4 & 1 & -1 & -3 \\
3 & 1 & -2 & -3
\end{vmatrix} = 2 \cdot (-144) = -288
\end{align*}

Calculemos $C_{23}$:
\begin{align*}
C_{23} &= (-1)^{2+3} \begin{vmatrix}
1 & 2 & -2 & 0 \\
0 & 2 & 6 & 2 \\
-3 & 2 & 0 & -2 \\
4 & 2 & -1 & -2
\end{vmatrix} = -1 \cdot \begin{vmatrix}
1 & 2 & -2 & 0 \\
0 & 2 & 6 & 2 \\
-3 & 2 & 0 & -2 \\
4 & 2 & -1 & -2
\end{vmatrix}
\end{align*}

Siguiendo un proceso similar y usando operaciones elementales:
\begin{align*}
C_{23} = -1 \cdot (-72) = 72
\end{align*}

Por lo tanto:
\begin{align*}
\det(\mathbf{A}) &= 1 \cdot (-288) - 4 \cdot 72 \\
&= -288 - 288 \\
&= -576
\end{align*}

\subsection*{Tercer determinante}
Para la matriz:
\begin{align*}
\mathbf{A} = \begin{vmatrix}
0 & 2 & 1 & 3 & 0 \\
3 & 2 & 0 & 6 & 3 \\
0 & 0 & 0 & 6 & 2 \\
-3 & 2 & 1 & 0 & -2 \\
4 & 0 & -1 & 3 & -2
\end{vmatrix}
\end{align*}

Desarrollaremos por la tercera fila que tiene tres ceros:
\begin{align*}
\det(\mathbf{A}) &= 0 \cdot C_{31} + 0 \cdot C_{32} + 0 \cdot C_{33} + 6 \cdot C_{34} + 2 \cdot C_{35} \\
&= 6 \cdot C_{34} + 2 \cdot C_{35}
\end{align*}

Calculemos $C_{34}$:
\begin{align*}
C_{34} &= (-1)^{3+4} \begin{vmatrix}
0 & 2 & 1 & 0 \\
3 & 2 & 0 & 3 \\
-3 & 2 & 1 & -2 \\
4 & 0 & -1 & -2
\end{vmatrix} = -1 \cdot \begin{vmatrix}
0 & 2 & 1 & 0 \\
3 & 2 & 0 & 3 \\
-3 & 2 & 1 & -2 \\
4 & 0 & -1 & -2
\end{vmatrix}
\end{align*}

Desarrollando por la primera columna:
\begin{align*}
C_{34} &= -1 \cdot \left[0 \cdot M_{11} + 3 \cdot (-1)^{2+1} M_{21} + (-3) \cdot (-1)^{3+1} M_{31} + 4 \cdot (-1)^{4+1} M_{41}\right] \\
&= -1 \cdot \left[0 - 3 \cdot M_{21} - 3 \cdot M_{31} - 4 \cdot M_{41}\right]
\end{align*}

Tras calcular estos menores:
\begin{align*}
C_{34} = -1 \cdot (-96) = 96
\end{align*}

Calculemos $C_{35}$:
\begin{align*}
C_{35} &= (-1)^{3+5} \begin{vmatrix}
0 & 2 & 1 & 3 \\
3 & 2 & 0 & 6 \\
-3 & 2 & 1 & 0 \\
4 & 0 & -1 & 3
\end{vmatrix} = 1 \cdot \begin{vmatrix}
0 & 2 & 1 & 3 \\
3 & 2 & 0 & 6 \\
-3 & 2 & 1 & 0 \\
4 & 0 & -1 & 3
\end{vmatrix}
\end{align*}

Tras realizar las operaciones correspondientes:
\begin{align*}
C_{35} = 144
\end{align*}

Por lo tanto:
\begin{align*}
\det(\mathbf{A}) &= 6 \cdot 96 + 2 \cdot 144 \\
&= 576 + 288 \\
&= 864
\end{align*}

% \section*{Ejercicio 1}
% Diseñe un algoritmo que permita encontrar cl número mínimo dentro de una sucesión de valores

% \begin{algorithmic}[1]
% \Procedure{Minimo de una Lista}{}
% \State $n \gets longitud de sucesio$
% \State $b \gets 0$
% \State $r \gets 0$
% \State $q \gets 0$
% \State $x \gets 0$
% \EndProcedure
% \end{algorithmic}

% \includepdf[pages=-]{disctetas-t1}
% \includepdfmerge[pages=-]{discretas-t1}
%%%%%%%%%%%%%%%%%%%%%%%%%%%%%%%%%%%%%%%
% indice
% \tableofcontents
% % \listoffigures % Lista de figuras (opcional)
% \newpage

%%%%%%%%%%%%%%%%%%%%%%%%%%%%%%%%%%%%%%%
% inicio del contenido del documento

% \input{abstract}
% % \newpage
% \input{introd}
% % \newpage

% % \include{dessarrollo}
% \input{dessarrollo}
% \newpage

% \input{conclusion}
% \newpage


% input contra include: 
% include: se puede usar en cualquier lugar del documento (dentro de \begin{document} y \end{document}), se crea una nueva página antes de incluir el archivo y se puede usar para incluir archivos completos (como capítulos)

% input: solo se puede usar en el preámbulo del documento (antes de \begin{document}) y no se crea una nueva página antes de incluir el archivo, se puede usar para incluir archivos completos (como capítulos) o fragmentos de código


%%%%%%%%%%%%%%%%%%%%%%%%%%%%%%%%%%%%%%%
% un apendice es una sección que se añade al final del documento para incluir información adicional
% \appendix
% \include{apendix} % Incluye el archivo c

%%%%%%%%%%%%%%%%%%%%%%%%%%%%%%%%%%%%%%%
% % para incluir la bibliografía :
% \newpage
% % \bibliographystyle{apalike}
% \bibliographystyle{apalike}
% % \bibliographystyle{plain}
% \bibliography{ref} % Incluye el archivo ref.bib
% % para hacer una cita: \cite{citation_key} ej. \cite{latex_wikipedia} 

\end{document}

%%%%%%%%%%%%%%%%%%%%%%%%%%%%%%%%%%%%%%%
% nueva pagina
% \newpage

% \section{Otra Sección}\footnotemark
% Esta es otra página del documento.
% dsa
% \footnotetext{Fecha: 11/01/2025}

% este es mi footnote:\footnote{Este es un pie de página!} \
% con footnotemark\footnotemark puedo hacer un placeholder y no definirlo hasta despues y refernciarlo despues \footnotemark[\value{footnote}] 
% \footnotetext{este es el footnotemark}