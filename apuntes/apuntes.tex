\documentclass{report} % clase de documento

% paquetes necesarios
\usepackage{fancyhdr} % Encabezado y pie de página
\usepackage{geometry} % Márgenes del documento
\usepackage[utf8]{inputenc} % Codificación de caracteres
\usepackage[spanish]{babel} % Idioma del documento
\usepackage{amsmath} % Paquete para escribir fórmulas matemáticas
\usepackage{graphicx} % Paquete para incluir imágenes
\usepackage{tikz} % Paquete para dibujar en LaTeX
\usepackage[colorlinks, linkcolor=black]{hyperref} % Paquete para hipervínculos
% \usepackage{lipsum} % Paquete para generar texto de ejemplo

% configuración de los hipervínculos del documento (colores)
\hypersetup{
    colorlinks=true, % false: boxed links; true: colored links
    linkcolor=black, % color of internal links
    citecolor=black, % color of links to bibliography
    filecolor=black, % color of file links
    urlcolor=black   % color of external links
}

\geometry{margin=1in} % Márgenes de las páginas

% configuración de los encabezados y pies de página
\pagestyle{fancy} % Encabezado y pie de página
\fancyhead{} % Limpia el encabezado
\fancyhead[L]{Leonardo Daniel Aviña Neri} % Encabezado a la izqui
\fancyhead[R]{UNIVERSIDAD DE GUANAJUATO} % Encabezado a la derecha

\begin{document} % Inicio del documento

\begin{titlepage} % Portada

    % \begin{tikzpicture}[remember picture, overlay]
    %     \fill[blue!10,opacity=0.5] (current page.north west) rectangle ([yshift=-\paperheight/1]current page.north east);
    % \end{tikzpicture}

    \begin{flushleft}
        \includegraphics[width=0.4\textwidth]{../img/logo_ug.png} % Logo de la universidad
    \end{flushleft}

    \begin{flushright}
        \vspace{-1.6cm}
        \textbf{UNIVERSIDAD DE GUANAJUATO}
    \end{flushright}

    \vspace{2cm}

    \begin{center}
        \textbf{\Huge APUNTES ÁLGEBRA LINEAL
        } % Título en negritas y muy grande
    \end{center}

    \vspace{2cm}
    
    \begin{flushleft}
        \hspace*{4cm}
        \textbf{UDA: }Álgebra lineal \\
        \hspace*{4cm}
        \textbf{DOCENTE: }Ing Francisco Javier Gonzalez Martinez \\
        \hspace*{4cm}
        \textbf{ALUMNO: }Leonardo Daniel Aviña Neri \\
        \hspace*{4cm}
        \textbf{FECHA: } 11/01/2025 \\
        \hspace*{4cm}
        \textbf{CARRERA:} Licenciatura En Ingeniería De Datos \\
        \hspace*{4cm}
        E Inteligencia Artificial (LIDIA) \\
        \hspace*{4cm}
        \textbf{CLAVE UDA:} NELI06001
    \end{flushleft}

    \vfill

    \begin{center}
        \includegraphics[width=0.4\textwidth]{../img/logo_grande.jpg} \\ % Imagen centrada al final de la hoja
        \vspace{0.3cm}
        Semestre Enero-Junio \the\year \
        Campus Irapuato-Salamanca
    \end{center}
\end{titlepage}

%%%%%%%%%%%%%%%%%%%%%%%%%%%%%%%%%%%%%%%
% indice
\tableofcontents
% \listoffigures % Lista de figuras (opcional)
\newpage

%%%%%%%%%%%%%%%%%%%%%%%%%%%%%%%%%%%%%%%
% inicio del contenido del documento

% \include{ch-dedicatoria} % Incluye el archivo ch-dedicatoria.tex
% \newpage

% parcial 1
\chapter{Parcial 1} % portada del parcial 1
\section*{Temas}
% PONER TEMARIO
\newpage

%%%%%%%%%%%%%        APUNTES     %%%%%%%%%%%%%%
\hspace*{-0.64cm}
Fecha: 11/03/2025

\section{Examen 1}

\begin{equation}
\begin{aligned}
y - 2z &= -5 \\
2x - y + z &= -2 \\
4x - y &= -4
\end{aligned}
\end{equation}

\begin{equation}
    \begin{aligned}
    x - 2y + 3z -w &= 10 \\
    -x + y - z +2w &= 2 \\
    3x + 5y -2z +3w &= -9 \\
    -2x - y +5z +3w &= 3 
    \end{aligned}
    \end{equation}

\newpage


% parcial 2
\chapter{Parcial 2} % portada del parcial 2
\section*{Temas}
% PONER TEMARIO
\newpage


%%%%%%%%%%%%%        APUNTES     %%%%%%%%%%%%%%
% \hspace*{-0.64cm}
Fecha: 11/03/2025

\section{Examen 1}

\begin{equation}
\begin{aligned}
y - 2z &= -5 \\
2x - y + z &= -2 \\
4x - y &= -4
\end{aligned}
\end{equation}

\begin{equation}
    \begin{aligned}
    x - 2y + 3z -w &= 10 \\
    -x + y - z +2w &= 2 \\
    3x + 5y -2z +3w &= -9 \\
    -2x - y +5z +3w &= 3 
    \end{aligned}
    \end{equation}

% \newpage

% parcial 3
\chapter{Parcial 3} % portada del parcial 3
\section*{Temas}
% PONER TEMARIO
\newpage

%%%%%%%%%%%%%        APUNTES     %%%%%%%%%%%%%%
% \hspace*{-0.64cm}
Fecha: 11/03/2025

\section{Examen 1}

\begin{equation}
\begin{aligned}
y - 2z &= -5 \\
2x - y + z &= -2 \\
4x - y &= -4
\end{aligned}
\end{equation}

\begin{equation}
    \begin{aligned}
    x - 2y + 3z -w &= 10 \\
    -x + y - z +2w &= 2 \\
    3x + 5y -2z +3w &= -9 \\
    -2x - y +5z +3w &= 3 
    \end{aligned}
    \end{equation}

% \newpage



%%%%%%%%%%%%%%%%%%%%%%%%%%%%%%%%%%%%%%%
% un apendice es una sección que se añade al final del documento para incluir información adicional
% \appendix
% \include{apendix} % Incluye el archivo c

%%%%%%%%%%%%%%%%%%%%%%%%%%%%%%%%%%%%%%%
% para incluir la bibliografía :
% \newpage
% \bibliographystyle{apalike}
% \bibliography{ref} % Incluye el archivo ref.bib
% para hacer una cita: \cite{citation_key} ej. \cite{latex_wikipedia} 

\end{document}

%%%%%%%%%%%%%%%%%%%%%%%%%%%%%%%%%%%%%%%
% nueva pagina
% \newpage

% \section{Otra Sección}\footnotemark
% Esta es otra página del documento.
% dsa
% \footnotetext{Fecha: 11/01/2025}

% este es mi footnote:\footnote{Este es un pie de página!} \
% con footnotemark\footnotemark puedo hacer un placeholder y no definirlo hasta despues y refernciarlo despues \footnotemark[\value{footnote}] 
% \footnotetext{este es el footnotemark}